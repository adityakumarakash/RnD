\chapter{Experiments}
We give examples to show how PPCA can be exploited for practical examples. The experiments focus on the application of PPCA to the dataset with missing values.


\section{Dataset}
We use following dataset :
\begin{enumerate}
\item \textbf{Tobamovirus} dataset : 38 virus , each with 18 features
\item \textbf{MNIST} dataset : The MNIST database of handwritten digits has a training set of 60,000 examples, and a test set of 10,000 examples. The digits have been size-normalized and centered in a fixed-size image of 28$\times$28 pixels. 
\item \textbf{USPS} dataset : Handwritten Digits, 8-bit grayscale images of "0" through "9"; 1100 examples of each class. 
\item \textbf{Binary Alphadigits} : Binary 20x16 digits of "0" through "9" and capital "A" through "Z". 39 examples of each class.
\end{enumerate}

\section{Experiment Design}
\begin{itemize}
\item For the \textbf{Tobamovirus} dataset, the data is projected into 2 dimensions for the purpose of visualization of dimension reduction by PCA and PPCA. The dataset is claimed to have three sub-groups. The missing data is simulated by randomly removing each value in the dataset with probability 20\%. The aim is to find how much of the sub-groups is being preserved. 
\item For the handwritten digits datasets, the data was randomly divided intp 7:3 ratio for training and testing, in case the two sets are not present. \\\\
We train a \emph{classifier based on mahalanobis distance}. For each digit a factor matrix ($\mathbf{W}$) is obtained using PCA/PPCA on the training data. Based on the factor matrix, we find the projections of all the training data points. Mahanalobis distance of each test data sample is calculated in latent dimension from the training set of each digit. The digit which gives least distance is predicted as the label.\\\\
For missing data case, the factor matrix and latent variables are learnt from training data having missing values. The missing values are simulated by randomly removing the data with a given probability. The prediction is done using these learnt values. The algorithms are analysed for different amount of missing values.\\\\
With this experiment, we try to find the behaviour (prediction accuracy) of each of the algorithms with different amount of training data.
\end{itemize}
