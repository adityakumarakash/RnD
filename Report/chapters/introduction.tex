\chapter{Introduction}

Principal Component analysis (PCA) is a ubiquitous tool for dimensionality reduction.  It has many applications data compression, visualization, image processing, exploratory data analysis, pattern recognition and time series prediction. \\
There are a number of optimization criteria to derive PCA. The most important of these is in terms of stardardized linear projection which maximizes the variance in the projected space \cite{PCA}. Assume that $\{y_i\}, i \in \{1, 2, ..., n\}$ is a set of $d$ dimensional data vectors. Then the $k$ principal axes $\{w_j\}, j \in \{1, 2, ..., k\}$ are those orthogonal axes onto which the retained variance under projection is maximal. It can be shown that $w_j$ are given by $k$ dominant eigen vectors (those with largest eigen values, $\lambda_j$) of the sample covariance matrix  
\begin{center}
\begin{equation}
	\mathbf{S = \sum_{i=1}^n (y_i - \bar{y})(y_i - \bar{y})^T}
\end{equation}
\end{center}
where $\mathbf{\bar{y}}$ is the sample mean, such that
\begin{equation}
\mathbf{Sw_j=\lambda_jw_j}
\end{equation}
The $k$ principal components of the observed vector $\mathbf{y_i}$ are given by the vector 
\begin{equation}
\mathbf{x_i=W^T(y_i - \bar{y})}
\end{equation}
where $\mathbf{W = (w_1;w_2;...;w_k)}$. The variables $\mathbf{x_i}$ are uncorrelated such that the covariance matrix $\sum_{i=1}^n\frac{\mathbf{x_ix_i^T}}{n}$ is diagonal with elements $lambda_i$.\\
A complementary property of PCA is that of all the orthogonal linear projections, the principal component minimizes the squared reconstruction error. However, PCA does not provide a probabilistic model of data. This gives motivation for PPCA.

\section{Motivation}
A probabilistic formulation of PCA from a Gaussian latent variable model is obtained. This is closely related to factor analysis. PCA could be viewed a limiting case of such a Gaussian model. In such a formulation, the principal axes emerge as maximum likelihood parameter estimates. Such a probabilistic formulation is intuitively applealing, as the definition of a likelihood measure enables comparison with other probabilistic techniques, while facilitating statistical testing and permitting the application of Bayesian methods. Further motivation behind a probabilistic PCA is that it conveys additional practical advantage as :
\begin{itemize}
\item The probability model offers the potential to extend the scope of conventional PCA, such as using probabilistic mixtures and PCA projections in missing data case.
\item PPCA can be utilized as a general Gaussian density model. This allows the maximum likelihood estimates for the parameters associated with the covariance matrix to be efficiently computed from the data principal components.
\end{itemize}

