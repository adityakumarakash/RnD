\chapter{Conclusion}
In this work, we see how principal component analysis model can be viewed as a maximum likelihood procedure based on a probability density model of the observed data. We also see links to PPCA to factor analysis and subtle difference between the two, as well as the underlying assumption for PPCA. An EM algorithm was discussed for PPCA which iteratively maximized the likelihood of the data.\\\\
The main aim of this work was understanding PPCA and its application on real world dataset. We apply PPCA to the case of missing data and compare its performance with different version of PCA made to handle missing data. In the \emph{Results} chapter we see how PPCA is capable of handling missing data along with providing a reasonable model for interpretation of the results. But we found out that for the datasets which we covered PCA with missing data replaced by mean of non-missing data gives best performance. This leads us to conclude that for cases where data is not generated using mixtures of gaussian, the sophistication of PPCA is only to provide a probabilistic touch to the classical version of PCA.\\
The importance for EM algorithm of PPCA is more when the data has an inherent mixture model distribution. In such cases, PCA cannot trivially handle it. But PPCA could easily incorporate it into the model and handle such cases.